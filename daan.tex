% BabelvisCV LaTeX class
% Copyright © 2025-2025 Bas Magre <bas.magre@babelvis.nl>
% Licensed under the 3-Clause BSD License. See LICENSE file for details.
%
% Attributions
% ------------
% * babelviscv is based on the fortysecondscv class by René Wirnata
%   (rene.wirnata@pandascience.net), released under the 3-Clause BSD License and available under
%   https://github.com/PandaScience/FortySecondsCV
% * fortysecondscv is based on the twentysecondcv class by Carmine Spagnuolo
%   (cspagnuolo@unisa.it), released under the MIT license and available under
%   https://github.com/spagnuolocarmine/TwentySecondsCurriculumVitae-LaTex
% * further attributions are indicated immediately before corresponding code


%-------------------------------------------------------------------------------
%                             ADDITIONAL PACKAGES
%-------------------------------------------------------------------------------
\documentclass[
	a4paper
]{babelviscv}

% fine tune line spacing
% \usepackage{setspace}
% \setstretch{1.1}

% improve word spacing and hyphenation
\usepackage{microtype}
\usepackage{ragged2e}
\usepackage{enumitem}

% uncomment in case you don't want any hyphenation
% \usepackage[none]{hyphenat}

% take care of proper font encoding
\ifxetexorluatex
	\usepackage{fontspec}
	\defaultfontfeatures{Ligatures=TeX}
	% \newfontfamily\headingfont[Path=fonts/]{segoeuib.ttf} % use local font
\else
	\usepackage[utf8]{inputenc}
	\usepackage[T1]{fontenc}
\fi

% use a sans serif font as default
% \usepackage[sfdefault]{ClearSans}
% \usepackage[sfdefault]{noto}
% \usepackage{fontenc}
\usepackage[sfdefault]{carlito}

% multi-language CV XeLaTeX and polyglossia (should also work with LuaLaTeX)
% NOTE: breaks \pointskill, \membership and some spacings
% \ifxetexorluatex
% 	\usepackage{polyglossia}
% 	\newfontfamily\arabicfontsf[Script=Arabic,Scale=1.5]{Amiri}
% 	\newfontfamily\englishfontsf{Clear Sans}
% 	\setmainfont{Amiri}
% 	\setdefaultlanguage{arabic}
% 	\setotherlanguage{english}
% \fi

% enable mathematical syntax for some symbols like \varnothing
\usepackage{amssymb}

% bubble diagram configuration
\usepackage{smartdiagram}
\smartdiagramset{
	% default font size is \large, so adjust to harmonize with sidebar layout
	bubble center node font = \footnotesize,
	bubble node font = \footnotesize,
	% default: 4cm/2.5cm; make minimum diameter relative to sidebar size
	bubble center node size = 0.4\sidebartextwidth,
	bubble node size = 0.25\sidebartextwidth,
	distance center/other bubbles = 1.5em,
	% set center bubble color
	bubble center node color = maincolor!70,
	% define the list of colors usable in the diagram
	set color list = {maincolor!10, maincolor!40,
	maincolor!20, maincolor!60, maincolor!35},
	% sets the opacity at which the bubbles are shown
	bubble fill opacity = 0.8,
}

%-------------------------------------------------------------------------------
%                            PERSONAL INFORMATION
%-------------------------------------------------------------------------------
%% mandatory information
% your name
\cvname{Daan Roeterink}

%% optional information
% profile picture
\cvprofilepic{pics/profile/daan.jpg}
% NOTE: ordering in sidebar will mimic the following order
\cvnameprofile{Daniël Willem Roeterink (Daan)}
% email address
\cvmail{daan.roeterink@babelvis.nl}
% phone number
\cvphone{+31 6 10 40 91 96}
% short address/location, use \newline if more than 1 line is required
\cvaddress{Zutphenseweg 23 GU 4.11\newline 7418 AG Deventer}
% date of birth
\cvbirthday{10 februari 1990}
% personal website
\cvsite{https://babelvis.nl}
% any other custom entry
\cvcustomdata{\faFlag}{Nederlands}

% social network accounts incl. proper hyperlinks
\sncustomdata{\faLinkedin}{www.linkedin.com/in/daan-roeterink/}{LinkedIn}
\sncustomdata{\faGithub}{https://github.com/daanroeterink}{Github (Privé projecten)}

%-------------------------------------------------------------------------------
%                              SIDEBAR 1st PAGE
%-------------------------------------------------------------------------------
% add more profile sections to sidebar on first page

\addtofrontsidebar{

	\sidesection{Competenties}
		\babelviscompetentie{Samenwerkingsgericht}
		\babelviscompetentie{Loyaal en betrouwbaar}
		\babelviscompetentie{Leergierig en nieuwsgierig}
		\babelviscompetentie{Oplossingsgericht}
		\babelviscompetentie{Verbindend vermogen}
}


%-------------------------------------------------------------------------------
%                              SIDEBAR 2nd PAGE
%-------------------------------------------------------------------------------
\definecolor{pastelgreen}{HTML}{D7ECD9}
\definecolor{pastelpurple}{HTML}{D5D6EA}
\definecolor{pastelorange}{HTML}{F5D5CB}
\definecolor{pastelyellow}{HTML}{F6F6EB}

\addtobacksidebar{

	% include gosquare national flags from https://github.com/gosquared/flags;
	% naming according to ISO 3166-1 alpha-2 country codes
	\graphicspath{{pics/gosquared-flags/flags/flags-iso/shiny/64}}

	\sidesection{Vaardigheden}
		\babelvisskill{C\#}{5}
		\babelvisskill{Pipelines/CICD}{5}
		\babelvisskill{Docker}{5}
		\babelvisskill{Kubernetes}{5}
		\babelvisskill{Helm}{4}
		\babelvisskill{MSSQL}{4}
		\babelvisskill{Python}{3}
		\babelvisskill{Ansible}{3}
		\babelvisskill{Java}{2}

	\sidesection{Talen}
		\taalkill{\flag{NL.png}}{Nederlands}{Moedertaal}
		\taalkill{\flag{GB.png}}{Engels}{Goed}
		% \pointskill{\flag{DE.png}}{Duits}{2}

	% \sidesection{Levensmotto}
	% 	\aboutme{
	% 		Er zijn geen problemen er zijn alleen uitdagingen.
	% 	}
}


%-------------------------------------------------------------------------------
%                         TABLE ENTRIES RIGHT COLUMN
%-------------------------------------------------------------------------------
\begin{document}

\makefrontsidebar

\cvsection{ } % hack zodat hij wat later begint
\cvsection{Profiel}
\sidetext{Ik ben een gedreven Cloud- en DevOps Engineer met diepgaande kennis van Kubernetes en moderne ontwikkel- en deploymentomgevingen. Dankzij mijn achtergrond als programmeur en mijn ervaring in DevOps weet ik complexe technische vraagstukken snel te doorgronden en om te zetten in praktische oplossingen. 
\newline\newline  
Ik ben een echte teamplayer: ik verbind mensen en teams door hulp te bieden waar nodig en zorg dat samenwerking soepel verloopt. Ik ben loyaal, gedreven en altijd bereid om een stap extra te zetten. Door mijn combinatie van technische expertise en verbindend vermogen help ik teams niet alleen succesvol projecten te realiseren, maar ook structureel beter samen te werken.
}

\cvsection{Werkervaring}
\cvsubsection{Babelvis}
	\begin{cvtable}[0.5]
		\cvitem{2025 -- heden}{Co-founder}{}
			{Adviseren en realiseren van veilige cloud- en DevOps-oplossingen met focus op Europese soevereine cloud.}
	\end{cvtable}
\cvsubsection{Topicus}
	\begin{cvtable}[0.5]
		\cvitem{2022 -- 2025}{(Azure) Cloud Engineer}{}
			{Migratie van ons Force Product Suite naar Azure (Cloud)}
			\cvitemlist{Key Results:}{Realiseren CICD Pipeline in Azure, Containerization van 20+ applicaties, Kubernetes deployments met behulp van Helm Charts, Onboarden 20+ Teams doormiddel van workshops}
		\cvitem{2020 -- 2022}{DevOps/Beheer}{}
			{Het geautomatiseerd opzetten van een acceptatie en productie omgeving voor een nieuwe klant. Centralisatie van configuratie voor de verschillende componenten in dit klantlandschap.}
			\cvitemlist{Key Results:}{Verschillende zelfgemaakte Ansible modules voor het configuren. Omgeving vanaf de eerste steen geautomatiseerd opgezet.}
		\cvitem{2016 -- 2020}{Software engineer}{}
			{Gewerkt aan verschillende financiele producten. backend en frontend. \newline Hier werkte ik met C\# (MVC/ASP.NET/NHibernate/Spring/Autofac en meer), Python, Jenkins, mssql, Angular, Git, RabbitMq, MSMQ, Ansible, Docker.}
	\end{cvtable}


\newpage
\makebacksidebar

\cvsection{Opleidingen}
	\begin{cvtable}[1.5]
		\cvitem{2011 -- 2016}{HBO Saxion}{Enschede}
			{Informatica}
		\cvitem{2009 -- 2011}{MBO-4 Graafschap College}{Doetinchem}
			{Applicatie ontwikkelaar}
		\cvitem{2007 -- 2009}{MBO-3 Graafschap College}{Doetinchem}
			{Medewerker beheer ICT}
	\end{cvtable}

\cvsection{Cursussen}
	\begin{cvtable}[1.5]
		\cvitem{2024}{Senior craftsmanship}{}
			{Hoe om te gaan met moderne functies in C\# en hoe de organisaties zover zien te krijgen dat deze kunnen worden toegevoegd in bestaande code-base.}
   		\cvitem{2022}{Situationeel leiderschap}{}
			{Training omtrent zelfleiderschap, situationeel leiderschap en het toepassen van het SLII model.}
		\cvitem{2020}{Sierksma training}{}
			{Gericht op het verbeteren van communicatie, persoonlijke effectiviteit en de effectiviteit binnen een organisatie.}
		\cvitem{2017}{Zakelijk engels}{}
			{Focus op het gebruik van Engels in een zakelijke omgeving.}
	\end{cvtable}

\cvsection{Nevenactiviteiten}
	\cvsubsection{Bestuur Huis van de Wijk}
		\begin{cvtable}[1.5]
			\cvitem{2023 -- heden}{Bestuurslid}{}
				{Bestuur en Technische ondersteuning voor de verschillende zaken die allemaal spelen in een buurthuis}
		\end{cvtable}


\cvsignature

\end{document}
