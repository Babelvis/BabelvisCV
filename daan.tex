% FortySecondsCV LaTeX template
% Copyright © 2019-2022 René Wirnata <rene.wirnata@pandascience.net>
% Licensed under the 3-Clause BSD License. See LICENSE file for details.
%
% Please visit https://github.com/PandaScience/FortySecondsCV for the most
% recent version! For bugs or feature requests, please open a new issue on
% github.
%
% Contributors:
% https://github.com/PandaScience/FortySecondsCV/graphs/contributors
%
% Attributions
% ------------
% * fortysecondscv is based on the twentysecondcv class by Carmine Spagnuolo
%   (cspagnuolo@unisa.it), released under the MIT license and available under
%   https://github.com/spagnuolocarmine/TwentySecondsCurriculumVitae-LaTex
% * further attributions are indicated immediately before corresponding code


%-------------------------------------------------------------------------------
%                             ADDITIONAL PACKAGES
%-------------------------------------------------------------------------------
\documentclass[
	a4paper,
	nameandjobposition=before,
	profilepicstyle=profilecircle
]{babelviscv}

% fine tune line spacing
% \usepackage{setspace}
% \setstretch{1.1}

% improve word spacing and hyphenation
\usepackage{microtype}
\usepackage{ragged2e}
\usepackage{enumitem}

% uncomment in case you don't want any hyphenation
% \usepackage[none]{hyphenat}

% take care of proper font encoding
\ifxetexorluatex
	\usepackage{fontspec}
	\defaultfontfeatures{Ligatures=TeX}
	% \newfontfamily\headingfont[Path=fonts/]{segoeuib.ttf} % use local font
\else
	\usepackage[utf8]{inputenc}
	\usepackage[T1]{fontenc}
\fi

% use a sans serif font as default
% \usepackage[sfdefault]{ClearSans}
\usepackage[sfdefault]{noto}
% \usepackage{fontenc}

% multi-language CV XeLaTeX and polyglossia (should also work with LuaLaTeX)
% NOTE: breaks \pointskill, \membership and some spacings
% \ifxetexorluatex
% 	\usepackage{polyglossia}
% 	\newfontfamily\arabicfontsf[Script=Arabic,Scale=1.5]{Amiri}
% 	\newfontfamily\englishfontsf{Clear Sans}
% 	\setmainfont{Amiri}
% 	\setdefaultlanguage{arabic}
% 	\setotherlanguage{english}
% \fi

% enable mathematical syntax for some symbols like \varnothing
\usepackage{amssymb}

% bubble diagram configuration
\usepackage{smartdiagram}
\smartdiagramset{
	% default font size is \large, so adjust to harmonize with sidebar layout
	bubble center node font = \footnotesize,
	bubble node font = \footnotesize,
	% default: 4cm/2.5cm; make minimum diameter relative to sidebar size
	bubble center node size = 0.4\sidebartextwidth,
	bubble node size = 0.25\sidebartextwidth,
	distance center/other bubbles = 1.5em,
	% set center bubble color
	bubble center node color = maincolor!70,
	% define the list of colors usable in the diagram
	set color list = {maincolor!10, maincolor!40,
	maincolor!20, maincolor!60, maincolor!35},
	% sets the opacity at which the bubbles are shown
	bubble fill opacity = 0.8,
}

%-------------------------------------------------------------------------------
%                            PERSONAL INFORMATION
%-------------------------------------------------------------------------------
%% mandatory information
% your name
\cvname{Daan Roeterink}
% job title/career
% \cvjobtitle{Panda Scientist,\\[0.2em] Panda of the Year}

%% optional information
% profile picture
\cvprofilepic{pics/profile/daan.jpg}
% logo picture
\cvlogopic{pics/logo_babelvis.png}

% NOTE: ordering in sidebar will mimic the following order
\cvnameprofile{Daniël Willem Roeterink (Daan)}
% email address
\cvmail{daan.roeterink@babelvis.nl}
% phone number
\cvphone{+31 6 10 40 91 96}
% short address/location, use \newline if more than 1 line is required
\cvaddress{Zutphenseweg 23 GU 4.11\newline 7418 AG Deventer}
% date of birth
\cvbirthday{10 februari 1990}
% personal website
\cvsite{https://babelvis.nl}
% any other custom entry
\cvcustomdata{\faFlag}{Nederlands}

%-------------------------------------------------------------------------------
%                              SIDEBAR 1st PAGE
%-------------------------------------------------------------------------------
% add more profile sections to sidebar on first page
\addtofrontsidebar{
	\sidesection{Competenties}
		\babelviscompetentie{Behulpvaardig}
		\babelviscompetentie{Mensgericht}
		\babelviscompetentie{Analytisch}
		\babelviscompetentie{Samenwerken}
		\babelviscompetentie{Resultaatgericht}
}


%-------------------------------------------------------------------------------
%                              SIDEBAR 2nd PAGE
%-------------------------------------------------------------------------------
\definecolor{pastelgreen}{HTML}{D7ECD9}
\definecolor{pastelpurple}{HTML}{D5D6EA}
\definecolor{pastelorange}{HTML}{F5D5CB}
\definecolor{pastelyellow}{HTML}{F6F6EB}

\addtobacksidebar{

	% include gosquare national flags from https://github.com/gosquared/flags;
	% naming according to ISO 3166-1 alpha-2 country codes
	\graphicspath{{pics/gosquared-flags/flags/flags-iso/shiny/64}}

	\sidesection{Vaardigheden}
		\babelvisskill{C\#}{5}
		\babelvisskill{Pipelines/CICD}{5}
		\babelvisskill{Docker}{5}
		\babelvisskill{Kubernetes}{5}
		\babelvisskill{Helm}{4}
		\babelvisskill{MSSQL}{4}
		\babelvisskill{Python}{3}
		\babelvisskill{Ansible}{3}
		\babelvisskill{Java}{2}

	\sidesection{Talen}
		\taalkill{\flag{NL.png}}{Nederlands}{Moedertaal}
		\taalkill{\flag{GB.png}}{Engels}{Goed}
		% \pointskill{\flag{DE.png}}{Duits}{2}

	% \sidesection{Levensmotto}
	% 	\aboutme{
	% 		Er zijn geen problemen er zijn alleen uitdagingen.
	% 	}

	% social network accounts incl. proper hyperlinks
	\sidesection{Social Network}
		\begin{icontable}{1.7em}{0.4em}
		% see https://www.ipgp.fr/~moguilny/LaTeX/fontawesome5Icons.pdf
			\social{\faLinkedin}
				{https://www.linkedin.com/in/daan-roeterink-5283696b/}
				{LinkedIn}
			\social{\faGithub}
				{https://github.com/daanroeterink}
				{Github (Privé projecten)}
		\end{icontable}
}


%-------------------------------------------------------------------------------
%                         TABLE ENTRIES RIGHT COLUMN
%-------------------------------------------------------------------------------
\begin{document}

\makefrontsidebar

\cvsection{ } % hack zodat hij wat later begint
\cvsection{Profiel}
\sidetext{Ik ben een softwareontwikkelaar die zich de laatste jaren heeft gefocussed op DevOps. Begonnen met het eigen maken van Ansible en vervolgens doorontwikkeld in het gebruik en configureren van Containers en Kubernetes. 
\newline Werk het liefst in een team waar de samenwerking centraal staat.}

\cvsection{Werkervaring}
\cvsubsection{Babelvis}
	\begin{cvtable}[0.5]
		\cvitemnolocation{2025 -- heden}{Co-founder}
			{Het helpen van organisaties om hun migratie naar de (Europese) cloudproviders tot een succes te brengen}
	\end{cvtable}
\cvsubsection{Topicus}
	\begin{cvtable}[0.5]
		\cvitemnolocation{2022 -- 2025}{(Azure) Cloud Engineer}
			{Migratie van ons Force Product Suite naar Azure (Cloud)}
			\cvitemlist{Key Results:}{Realiseren CICD Pipeline in Azure, Containerization van 20+ applicaties, Kubernetes deployments met behulp van Helm Charts, Onboarden 20+ Teams doormiddel van workshops}
		\cvitemnolocation{2020 -- 2022}{DevOps/Beheer}
			{Het geautomatiseerd opzetten van een acceptatie en productie omgeving voor een nieuwe klant. Centralisatie van configuratie voor de verschillende componenten in dit klantlandschap.}
			\cvitemlist{Key Results:}{Verschillende zelfgemaakte Ansible modules voor het configuren. Omgeving vanaf de eerste steen geautomatiseerd opgezet.}
		\cvitemnolocation{2016 -- 2020}{Software engineer}
			{Gewerkt aan verschillende financiele producten. backend en frontend. \newline Hier werkte ik met C\# (MVC/ASP.NET/NHibernate/Spring/Autofac en meer), Python, Jenkins, mssql, Angular, Git, RabbitMq, MSMQ, Ansible, Docker.}
	\end{cvtable}


\newpage
\makebacksidebar

\cvsection{Opleidingen}
	\begin{cvtable}[1.5]
		\cvitem{2011 -- 2016}{HBO Saxion}{Enschede}
			{Informatica
			\newline(geslaagd in 2016)}
		\cvitem{2009 -- 2011}{MBO-4 Graafschap College}{Doetinchem}
			{Applicatie ontwikkelaar
			\newline(geslaagd in 2011)}
		\cvitem{2007 -- 2009}{MBO-3 Graafschap College}{Doetinchem}
			{Medewerker beheer ICT
			\newline(geslaagd in 2009)}
	\end{cvtable}

\cvsection{Cursussen}
	\begin{cvtable}[1.5]
		\cvitemnolocation{2024}{Senior craftsmanship}
			{Hoe om te gaan met moderne functies in C\# en hoe de organisaties zover zien te krijgen dat deze kunnen worden toegevoegd in bestaande code-base.}
		\cvitemnolocation{2017}{Zakelijk engels}
			{Focus op het gebruik van Engels in een zakelijke omgeving.}
	\end{cvtable}

\cvsection{Nevenactiviteiten}
	\cvsubsection{Bestuur Huis van de Wijk}
		\begin{cvtable}[1.5]
			\cvitemnolocation{2023 -- heden}{Bestuurslid}
				{Bestuur en Technische ondersteuning voor de verschillende zaken die allemaal spelen in een buurthuis}
		\end{cvtable}


\cvsignature

\end{document}
